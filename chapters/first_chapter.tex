\chapter{مقدمه}
جهت تسهیل استفاده از این تمپلیت لازم است، ابتدا ساختار کلی این فایل را مورد بررسی قرار دهیم. در فولدر اصلی با فایل‌های زیر مواجه می‌شویم. 
\begin{contract}
	\item \textbf{\lr{:auxilliary}}
	فایل‌های کمکی که شامل بخش‌های زیرند: 
	\begin{itemize}
		\item \textbf{\lr{:commands.tex}}
		این فایل حاوی تمامی بسته‌های فراخوانی شده، تنظیمات و دستورات تعریف شده می باشد. 
		\item \textbf{\lr{:foots.tex}}
		این فایل نیز جهت تعریف کردن پانویس‌ها و کلمات اختصاری تعبیه شده‌است. 
		\item \textbf{\lr{:symbols.tex}}
		این فایل حاوی تعریف‌های مربوط به علائم اختصاری و نماد‌ها می‌باشد. 
		\item \textbf{\lr{:front\_cover.tex}} 
		صفحه‌ی اول (شامل بسّم اللّه الرحمن الرحیم) و صفحه‌ی عنوان فارسی در این بخش قرار دارند. دانشجو باید وارد این فایل شده و در محل مربوطه عنوان پایان‌نامه، نام و نام خانوادگی دانشجو، استاد راهنما و بعلاوه تاریخ درج پایان نامه را وارد کند. 
		\item  \textbf{\lr{:back\_cover.tex}}
		صفحه‌ی عنوان لاتین در این فایل قرار دارد. مجدداً مشابه بالا باید تغییرات لازم به لاتین درج شود. 
	\end{itemize}
	\item \textbf{\lr{:chapters} }
	فصول اصلی پایان‌نامه که در اینجا تعداد چهار فصل در نظر گرفته شده‌است. بسته به نیاز این فصول متغیرند. این فولدر حاوی فایل‌های زیر می‌باشد. 
	\begin{itemize}
		\item \textbf{\lr{:dedication.tex}} 
		در این فایل متن تقدیم نوشته می‌شود. 
		\item \textbf{\lr{:thanks.tex}} 
		در این فایل متن تشکر و قدردانی نوشته می‌شود. 
		\item \textbf{\lr{:farsi\_abstract.tex}} 
		چکیده‌ی فارسی در این فایل نوشته می‌شود. 
		\item \textbf{\lr{:first\_chapter.tex}} 
		فصل اول
		\item \textbf{\lr{:second\_chapter.tex}} 
		فصل دوم
		\item \textbf{\lr{:third\_chapter.tex}} 
		فصل سوم
		\item \textbf{\lr{:final\_chapter.tex}} 
		فصل چهارم 
		\item \textbf{\lr{:english\_abstract.tex}}
		چکیده‌ی لاتین در این فایل نوشته می‌شود. 
	\end{itemize}
	\item \textbf{\lr{:images}}
	تمامی تصاویر مورد استفاده را در این فولدر قرار می‌دهیم. 
	\item \textbf{\lr{:document.tex}}
	 فایل اصلی که تمامی فولدر ها شامل فولدر فایل‌های کمکی، هر یک از فصول به ترتیب، واژه‌نامه و فهرست اختصارت و فهرست مراجع در آن فراخوانی شده‌اند. 
	\item \textbf{\lr{:references.bib}}
	این فایل حاوی تمامی مراجع فراخوانی شده در متن پایان‌نامه است. 
\end{contract}

توجه داشته باشید که برای فراخواندن هر کدام از فایل‌ها با فرمت 
\lr{.tex} 
در 
\lr{preamble}
از دستور 
\begin{latin} 
\noindent\verb*|\input{<name.tex>}|
\end{latin}
و برای سایر فصول به همراه پیوست از دستور 
\begin{latin} 
\noindent\verb*|\include{<name.tex>}|
\end{latin}
استفاده شده است. برای حذف/اضافه هر گونه فایل با فرمت 
\lr{.tex} 
از این پروژه می‌توان از دستورات بالا در فایل اصلی یعنی 
\lr{document.tex}
استفاده کرد.  برای ایجاد و چاپ فهرست منابع، واژه‌نامه و فهرست اختصارات نیز به ترتیب از دستورات زیر استفاده شده است. 
\begin{flushleft}
	\begin{latin} 
		\begin{verbatim}
			\printbib
			\printglossary
			\printacronyms
		\end{verbatim}
	\end{latin}
\end{flushleft}



\section{تنظیمات برنامه}
قبل از کامپایل کردن برنامه حتما توجه داشته باشید که باید موتور نرم‌‌افزار
\lr{XeLatex} 
انتخاب شود. همچنین لازم به ذکر است که این قالب در نرم‌افزار 
\lr{TexStudio} 
نوشته شده است. پیشنهاد می‌کنم به دلیل حفظ راست--چین بودن متون فارسی در بخش ادیتور و سهولت کار کردن با آن، شما هم از این نرم‌افزار رایگان که هر سه سیستم عامل 
\lr{Windows, Linux, macOS} 
را پیشتیبانی می کند، استفاده کنید.البته قطعا حق انتخاب با شماست.  برای دریافت این نرم‌افزار به سایت زیر بروید. 
\begin{latin}
	\noindent
	\href{https://www.texstudio.org/}{TeXstudio}
\end{latin}
اگر از  نرم‌افزار
\lr{TexStudio} 
 استفاده می‌کنید، برای انتخاب موتور 
\lr{XeLatex} 
ابتدا وارد منوی 
\lr{Options} 
شده و بخش 
\lr{Configure TeXstudio} 
را انتخاب کنید. سپس در پجره‌ی باز شده وارد قسمت 
\lr{Build} 
شده و در بخش 
\lr{Meta Commands}
گزینه‌ی
\lr{Default Compiler} 
را روی 
\lr{XeLatex} 
قرار دهید و در نهایت کلید 
\lr{Save} 
را انتخاب کنید. برای استفاده از این قالب فرض شده است که به دستورات ابتدایی
\lr{\LaTeX}
آشنایی دارید. در غیر صورت حتما ابتدا به یادگیری 
\lr{\LaTeX}
بپردازید. برای یادگیری 
\lr{\LaTeX}
مراجع بسیار زیادی وجود دارد، اما برای شروع می‌توانید سایت های زیر مراجعه کرده و آموزش‌ها را مطالعه کنید. 

\begin{latin}
	\noindent
	\href{https://www.overleaf.com/learn}{Overleaf}\\
    \href{https://en.wikibooks.org/wiki/LaTeX}{Wikibooks}\\
	\href{http://parsilatex.com}{ParsiLatex}
\end{latin}
یک مرجع بسیار مناسب به نام 
\textit{مقدمه‌ای نه چندان کوتاه بر لاتک}
برای یادگیری لاتک نوشته‌ی 
\linebreak
\lr{Tobias Oetiker} 
به ترجمه‌ی مهدی امید علی هم در سایت 
 \href{https://www.ctan.org/tex-archive/info/lshort/persian}{Ctan}
 قابل دریافت می‌باشد. 





\section{متن تقدیم، تشکر و چکیده}
برای نوشتن متن تقدیم و تشکر و چکیده وارد فولدر 
\lr{chapters} 
بشوید و در فایل مربوطه متن مورد نظر را قرار دهید. 

\section{وارد کردن نماد ریاضی در فهرست  نماد ها}
برای وارد کردن یک نماد ریاضی در فهرست علائم و نشانه ها ابتدا وارد فولدر 
\lr{auxilliary} 
شده و فایل 
\lr{foots.tex} 
را باز کرده و از دستور زیر استفاده می کنیم. 
\begin{latin}
	\noindent
	\verb*|\persiangloss{<name>}{<symbol>}|
\end{latin}
در بخش 
\lr{<name>}
نام نماد و در بخش 
\lr{<symbol>} 
باید نماد ریاضی را در میان 
\verb|$ $| 
قرار‍~دهیم. برای مثال اگر بخواهیم 
$\lambda$ 
را به عنوان نماد مقدار~ویژه معرفی کنیم در بخش 
\lr{<name>}
می‌نویسیم مقدار ویژه و در بخش 
\lr{<symbol>} 
قرار می‌دهیم
\verb|$\lambda$|.

\section{فونت و سایز فونت‌های مورد استفاده}
فونت مورد استفاده در تیتر‌ها و متن اصلی در جدول 
\ref{fonts description} 
درج شده‌اند. در صورت نیاز به فونت‌های دیگر باید مراحلی که در ادامه آن را بیان می‌کنیم، طی شود. ابتدا وارد فولد 
\lr{auxilliary}
شده، فایل 
\lr{commands.tex} 
را باز کرده و نام فونت‌های دیگر را در قسمت مربوطه (پس از فراخوانی بسته‌ی ژی~پرشین) تایپ کنید. این فونت‌ها باید حتما روی کامپیوتر شما نصب شده باشد. ترجیحاً فایل 
\lr{.ttf} 
فونت‌ها را نیز در فولدر اصلی کپی و پیست کنید. منظور از محل ذکر شده دستورات زیر است. 
\begin{latin} 
	\begin{verbatim}
		\settextfont[BoldFont={IRNazaninBold.ttf},
		ItalicFont={IRNazaninIranic.ttf}]{IRNazanin.ttf}
		\defpersianfont\Btitr{B Titr.ttf}
	\end{verbatim}
\end{latin} 
همچنین می توانید برای استفاده از فونت بی‌تیتر در هر قسمت از متن از دستور زیر استفاده کنید. 
\begin{latin}
	\begin{verbatim}
		{\Btitr{<text>}}
	\end{verbatim}
\end{latin}
 
\begin{table}[h]
	\centering
	\caption[فونت‌های استفاده شده و توضیحات آن]
	{فونت‌های استفاده شده و توضیحات آن. مورد سوم و چهارم باید توسط نویسنده‌ در صورت لزوم اعمال شود.}
	\label{fonts description}
	\begin{tabular}{|m{5cm}|m{8cm}|}
		 \hline
		 نام فونت 
		 &
		 محل استفاده 
		 \\ \hline
		\lr{B Titr.ttf}
		 & 
		 تیتر ها 
		 
		 \\ \hline
	    \lr{IRNazanin.ttf}
		& 
		متن اصلی 
	    \\ \hline
		
		\lr{IRNazaninBold.ttf}
		& 
		کلمات کلیدی، نام قضایا، تعریف‌ها و مثال‌ها، لیست‌های شماره‌دار و سایر موارد مورد نیاز 
	     \\ \hline
		\lr{IRNazaninIranic.ttf}
		&
		استفاده از یک اصطلاح و لغت خاص در اولین بار 
		\\ \hline
	\end{tabular} 
\end{table}




